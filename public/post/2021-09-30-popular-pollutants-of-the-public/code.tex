% Options for packages loaded elsewhere
\PassOptionsToPackage{unicode}{hyperref}
\PassOptionsToPackage{hyphens}{url}
%
\documentclass[
]{article}
\usepackage{lmodern}
\usepackage{amssymb,amsmath}
\usepackage{ifxetex,ifluatex}
\ifnum 0\ifxetex 1\fi\ifluatex 1\fi=0 % if pdftex
  \usepackage[T1]{fontenc}
  \usepackage[utf8]{inputenc}
  \usepackage{textcomp} % provide euro and other symbols
\else % if luatex or xetex
  \usepackage{unicode-math}
  \defaultfontfeatures{Scale=MatchLowercase}
  \defaultfontfeatures[\rmfamily]{Ligatures=TeX,Scale=1}
\fi
% Use upquote if available, for straight quotes in verbatim environments
\IfFileExists{upquote.sty}{\usepackage{upquote}}{}
\IfFileExists{microtype.sty}{% use microtype if available
  \usepackage[]{microtype}
  \UseMicrotypeSet[protrusion]{basicmath} % disable protrusion for tt fonts
}{}
\makeatletter
\@ifundefined{KOMAClassName}{% if non-KOMA class
  \IfFileExists{parskip.sty}{%
    \usepackage{parskip}
  }{% else
    \setlength{\parindent}{0pt}
    \setlength{\parskip}{6pt plus 2pt minus 1pt}}
}{% if KOMA class
  \KOMAoptions{parskip=half}}
\makeatother
\usepackage{xcolor}
\IfFileExists{xurl.sty}{\usepackage{xurl}}{} % add URL line breaks if available
\IfFileExists{bookmark.sty}{\usepackage{bookmark}}{\usepackage{hyperref}}
\hypersetup{
  pdftitle={code.R},
  pdfauthor={pedro},
  hidelinks,
  pdfcreator={LaTeX via pandoc}}
\urlstyle{same} % disable monospaced font for URLs
\usepackage[margin=1in]{geometry}
\usepackage{color}
\usepackage{fancyvrb}
\newcommand{\VerbBar}{|}
\newcommand{\VERB}{\Verb[commandchars=\\\{\}]}
\DefineVerbatimEnvironment{Highlighting}{Verbatim}{commandchars=\\\{\}}
% Add ',fontsize=\small' for more characters per line
\usepackage{framed}
\definecolor{shadecolor}{RGB}{248,248,248}
\newenvironment{Shaded}{\begin{snugshade}}{\end{snugshade}}
\newcommand{\AlertTok}[1]{\textcolor[rgb]{0.94,0.16,0.16}{#1}}
\newcommand{\AnnotationTok}[1]{\textcolor[rgb]{0.56,0.35,0.01}{\textbf{\textit{#1}}}}
\newcommand{\AttributeTok}[1]{\textcolor[rgb]{0.77,0.63,0.00}{#1}}
\newcommand{\BaseNTok}[1]{\textcolor[rgb]{0.00,0.00,0.81}{#1}}
\newcommand{\BuiltInTok}[1]{#1}
\newcommand{\CharTok}[1]{\textcolor[rgb]{0.31,0.60,0.02}{#1}}
\newcommand{\CommentTok}[1]{\textcolor[rgb]{0.56,0.35,0.01}{\textit{#1}}}
\newcommand{\CommentVarTok}[1]{\textcolor[rgb]{0.56,0.35,0.01}{\textbf{\textit{#1}}}}
\newcommand{\ConstantTok}[1]{\textcolor[rgb]{0.00,0.00,0.00}{#1}}
\newcommand{\ControlFlowTok}[1]{\textcolor[rgb]{0.13,0.29,0.53}{\textbf{#1}}}
\newcommand{\DataTypeTok}[1]{\textcolor[rgb]{0.13,0.29,0.53}{#1}}
\newcommand{\DecValTok}[1]{\textcolor[rgb]{0.00,0.00,0.81}{#1}}
\newcommand{\DocumentationTok}[1]{\textcolor[rgb]{0.56,0.35,0.01}{\textbf{\textit{#1}}}}
\newcommand{\ErrorTok}[1]{\textcolor[rgb]{0.64,0.00,0.00}{\textbf{#1}}}
\newcommand{\ExtensionTok}[1]{#1}
\newcommand{\FloatTok}[1]{\textcolor[rgb]{0.00,0.00,0.81}{#1}}
\newcommand{\FunctionTok}[1]{\textcolor[rgb]{0.00,0.00,0.00}{#1}}
\newcommand{\ImportTok}[1]{#1}
\newcommand{\InformationTok}[1]{\textcolor[rgb]{0.56,0.35,0.01}{\textbf{\textit{#1}}}}
\newcommand{\KeywordTok}[1]{\textcolor[rgb]{0.13,0.29,0.53}{\textbf{#1}}}
\newcommand{\NormalTok}[1]{#1}
\newcommand{\OperatorTok}[1]{\textcolor[rgb]{0.81,0.36,0.00}{\textbf{#1}}}
\newcommand{\OtherTok}[1]{\textcolor[rgb]{0.56,0.35,0.01}{#1}}
\newcommand{\PreprocessorTok}[1]{\textcolor[rgb]{0.56,0.35,0.01}{\textit{#1}}}
\newcommand{\RegionMarkerTok}[1]{#1}
\newcommand{\SpecialCharTok}[1]{\textcolor[rgb]{0.00,0.00,0.00}{#1}}
\newcommand{\SpecialStringTok}[1]{\textcolor[rgb]{0.31,0.60,0.02}{#1}}
\newcommand{\StringTok}[1]{\textcolor[rgb]{0.31,0.60,0.02}{#1}}
\newcommand{\VariableTok}[1]{\textcolor[rgb]{0.00,0.00,0.00}{#1}}
\newcommand{\VerbatimStringTok}[1]{\textcolor[rgb]{0.31,0.60,0.02}{#1}}
\newcommand{\WarningTok}[1]{\textcolor[rgb]{0.56,0.35,0.01}{\textbf{\textit{#1}}}}
\usepackage{graphicx}
\makeatletter
\def\maxwidth{\ifdim\Gin@nat@width>\linewidth\linewidth\else\Gin@nat@width\fi}
\def\maxheight{\ifdim\Gin@nat@height>\textheight\textheight\else\Gin@nat@height\fi}
\makeatother
% Scale images if necessary, so that they will not overflow the page
% margins by default, and it is still possible to overwrite the defaults
% using explicit options in \includegraphics[width, height, ...]{}
\setkeys{Gin}{width=\maxwidth,height=\maxheight,keepaspectratio}
% Set default figure placement to htbp
\makeatletter
\def\fps@figure{htbp}
\makeatother
\setlength{\emergencystretch}{3em} % prevent overfull lines
\providecommand{\tightlist}{%
  \setlength{\itemsep}{0pt}\setlength{\parskip}{0pt}}
\setcounter{secnumdepth}{-\maxdimen} % remove section numbering

\title{code.R}
\author{pedro}
\date{2022-01-10}

\begin{document}
\maketitle

\begin{Shaded}
\begin{Highlighting}[]
\CommentTok{\#\#\# setting the environment}
\KeywordTok{rm}\NormalTok{(}\DataTypeTok{list =} \KeywordTok{ls}\NormalTok{())}
\KeywordTok{require}\NormalTok{(tidyverse)}
\KeywordTok{require}\NormalTok{(glue)}
\KeywordTok{require}\NormalTok{(knitr)}


\KeywordTok{setwd}\NormalTok{(}\StringTok{"\textasciitilde{}/Portfolio/web{-}tea/content/post/2021{-}09{-}30{-}popular{-}pollutants{-}of{-}the{-}public/index\_files"}\NormalTok{)}
\NormalTok{files \textless{}{-}}\StringTok{ }\KeywordTok{list.files}\NormalTok{(}\DataTypeTok{pattern =} \StringTok{"*F.csv"}\NormalTok{)}
\NormalTok{all\_files \textless{}{-}}\StringTok{ }\KeywordTok{map}\NormalTok{(files, read\_csv)}
\end{Highlighting}
\end{Shaded}

\begin{verbatim}
## Rows: 85 Columns: 10
\end{verbatim}

\begin{verbatim}
## -- Column specification ----------------------------------------------------
## Delimiter: ","
## chr (8): product_barcode, product_label, product_size, brand_name, man...
## dbl (2): bottle_weight, bottle_count
\end{verbatim}

\begin{verbatim}
## 
## i Use `spec()` to retrieve the full column specification for this data.
## i Specify the column types or set `show_col_types = FALSE` to quiet this message.
\end{verbatim}

\begin{verbatim}
## Rows: 190 Columns: 10
\end{verbatim}

\begin{verbatim}
## -- Column specification ----------------------------------------------------
## Delimiter: ","
## chr (8): product_barcode, product_label, product_size, brand_name, man...
## dbl (2): bottle_weight, bottle_count
\end{verbatim}

\begin{verbatim}
## 
## i Use `spec()` to retrieve the full column specification for this data.
## i Specify the column types or set `show_col_types = FALSE` to quiet this message.
\end{verbatim}

\begin{verbatim}
## Rows: 240 Columns: 10
\end{verbatim}

\begin{verbatim}
## -- Column specification ----------------------------------------------------
## Delimiter: ","
## chr (8): product_barcode, product_label, product_size, brand_name, man...
## dbl (2): bottle_weight, bottle_count
\end{verbatim}

\begin{verbatim}
## 
## i Use `spec()` to retrieve the full column specification for this data.
## i Specify the column types or set `show_col_types = FALSE` to quiet this message.
\end{verbatim}

\begin{verbatim}
## Rows: 446 Columns: 10
\end{verbatim}

\begin{verbatim}
## -- Column specification ----------------------------------------------------
## Delimiter: ","
## chr (8): product_barcode, product_label, product_size, brand_name, man...
## dbl (2): bottle_weight, bottle_count
\end{verbatim}

\begin{verbatim}
## 
## i Use `spec()` to retrieve the full column specification for this data.
## i Specify the column types or set `show_col_types = FALSE` to quiet this message.
\end{verbatim}

\begin{verbatim}
## Rows: 599 Columns: 10
\end{verbatim}

\begin{verbatim}
## -- Column specification ----------------------------------------------------
## Delimiter: ","
## chr (8): product_barcode, product_label, product_size, brand_name, man...
## dbl (2): bottle_weight, bottle_count
\end{verbatim}

\begin{verbatim}
## 
## i Use `spec()` to retrieve the full column specification for this data.
## i Specify the column types or set `show_col_types = FALSE` to quiet this message.
\end{verbatim}

\begin{verbatim}
## Rows: 631 Columns: 10
\end{verbatim}

\begin{verbatim}
## -- Column specification ----------------------------------------------------
## Delimiter: ","
## chr (8): product_barcode, product_label, product_size, brand_name, man...
## dbl (2): bottle_weight, bottle_count
\end{verbatim}

\begin{verbatim}
## 
## i Use `spec()` to retrieve the full column specification for this data.
## i Specify the column types or set `show_col_types = FALSE` to quiet this message.
\end{verbatim}

\begin{verbatim}
## Rows: 793 Columns: 10
\end{verbatim}

\begin{verbatim}
## -- Column specification ----------------------------------------------------
## Delimiter: ","
## chr (8): product_barcode, product_label, product_size, brand_name, man...
## dbl (2): bottle_weight, bottle_count
\end{verbatim}

\begin{verbatim}
## 
## i Use `spec()` to retrieve the full column specification for this data.
## i Specify the column types or set `show_col_types = FALSE` to quiet this message.
\end{verbatim}

\begin{verbatim}
## Rows: 855 Columns: 10
\end{verbatim}

\begin{verbatim}
## -- Column specification ----------------------------------------------------
## Delimiter: ","
## chr (8): product_barcode, product_label, product_size, brand_name, man...
## dbl (2): bottle_weight, bottle_count
\end{verbatim}

\begin{verbatim}
## 
## i Use `spec()` to retrieve the full column specification for this data.
## i Specify the column types or set `show_col_types = FALSE` to quiet this message.
\end{verbatim}

\begin{verbatim}
## Rows: 900 Columns: 10
\end{verbatim}

\begin{verbatim}
## -- Column specification ----------------------------------------------------
## Delimiter: ","
## chr (8): product_barcode, product_label, product_size, brand_name, man...
## dbl (2): bottle_weight, bottle_count
\end{verbatim}

\begin{verbatim}
## 
## i Use `spec()` to retrieve the full column specification for this data.
## i Specify the column types or set `show_col_types = FALSE` to quiet this message.
\end{verbatim}

\begin{verbatim}
## Rows: 985 Columns: 10
\end{verbatim}

\begin{verbatim}
## -- Column specification ----------------------------------------------------
## Delimiter: ","
## chr (8): product_barcode, product_label, product_size, brand_name, man...
## dbl (2): bottle_weight, bottle_count
\end{verbatim}

\begin{verbatim}
## 
## i Use `spec()` to retrieve the full column specification for this data.
## i Specify the column types or set `show_col_types = FALSE` to quiet this message.
\end{verbatim}

\begin{Shaded}
\begin{Highlighting}[]
\KeywordTok{names}\NormalTok{(all\_files) \textless{}{-}}\StringTok{ }\NormalTok{month.abb[}\DecValTok{3}\OperatorTok{:}\DecValTok{12}\NormalTok{] }\CommentTok{\#to name the data according to month created which in our case is march to december}

\NormalTok{combined\_file \textless{}{-}}\StringTok{ }\KeywordTok{bind\_rows}\NormalTok{(all\_files, }\DataTypeTok{.id =} \StringTok{"batch"}\NormalTok{)}

\CommentTok{\#\# cleaning our dataset}

\NormalTok{df \textless{}{-}}\StringTok{ }\NormalTok{combined\_file }\OperatorTok{\%\textgreater{}\%}
\CommentTok{\#remove the bottle weight column}
\CommentTok{\#remove the url column}
\StringTok{  }\KeywordTok{select}\NormalTok{(}\OperatorTok{{-}}\NormalTok{bottle\_weight, }\OperatorTok{{-}}\NormalTok{data\_url, }\OperatorTok{{-}}\NormalTok{product\_label, }\OperatorTok{{-}}\NormalTok{product\_barcode)}

\CommentTok{\#convert batch, manufacturer\_name, manufacturer\_country, scan\_country into factors, months}
\NormalTok{df}\OperatorTok{$}\NormalTok{batch \textless{}{-}}\StringTok{ }\KeywordTok{as.factor}\NormalTok{(df}\OperatorTok{$}\NormalTok{batch)}
\NormalTok{df}\OperatorTok{$}\NormalTok{manufacturer\_country \textless{}{-}}\StringTok{ }\KeywordTok{as.factor}\NormalTok{(df}\OperatorTok{$}\NormalTok{manufacturer\_country)}
\NormalTok{df}\OperatorTok{$}\NormalTok{manufacturer\_name \textless{}{-}}\StringTok{ }\KeywordTok{as.factor}\NormalTok{(df}\OperatorTok{$}\NormalTok{manufacturer\_name)}
\NormalTok{df}\OperatorTok{$}\NormalTok{scan\_country \textless{}{-}}\StringTok{ }\KeywordTok{as.factor}\NormalTok{(df}\OperatorTok{$}\NormalTok{scan\_country)}
\NormalTok{df}\OperatorTok{$}\NormalTok{brand\_name \textless{}{-}}\StringTok{ }\KeywordTok{as.factor}\NormalTok{(df}\OperatorTok{$}\NormalTok{brand\_name)}

\CommentTok{\#parse product\_size for uniform measurements}
\NormalTok{number\_pattern \textless{}{-}}\StringTok{ "\^{}}\CharTok{\textbackslash{}\textbackslash{}}\StringTok{d}\CharTok{\textbackslash{}\textbackslash{}}\StringTok{s?}\CharTok{\textbackslash{}\textbackslash{}}\StringTok{.?}\CharTok{\textbackslash{}\textbackslash{}}\StringTok{d*"}
\NormalTok{unit\_pattern \textless{}{-}}\StringTok{ "}\CharTok{\textbackslash{}\textbackslash{}}\StringTok{D+$"}

\NormalTok{df1 \textless{}{-}}\StringTok{ }\NormalTok{df }\OperatorTok{\%\textgreater{}\%}
\StringTok{  }\KeywordTok{mutate}\NormalTok{(}\DataTypeTok{product\_size\_extracted =} \KeywordTok{str\_extract}\NormalTok{(product\_size, number\_pattern)) }\OperatorTok{\%\textgreater{}\%}
\StringTok{  }\KeywordTok{mutate}\NormalTok{(}\DataTypeTok{units =} \KeywordTok{str\_extract}\NormalTok{(product\_size, unit\_pattern)) }\OperatorTok{\%\textgreater{}\%}
\StringTok{  }\KeywordTok{drop\_na}\NormalTok{(units)}

\NormalTok{df1}\OperatorTok{$}\NormalTok{units \textless{}{-}}\StringTok{ }\KeywordTok{str\_trim}\NormalTok{(df1}\OperatorTok{$}\NormalTok{units)}
\NormalTok{df1}\OperatorTok{$}\NormalTok{product\_size\_extracted \textless{}{-}}\StringTok{ }\KeywordTok{as.numeric}\NormalTok{(}\KeywordTok{str\_trim}\NormalTok{(df1}\OperatorTok{$}\NormalTok{product\_size\_extracted))}

\NormalTok{df1}\OperatorTok{$}\NormalTok{units \textless{}{-}}\StringTok{ }\KeywordTok{ifelse}\NormalTok{(df1}\OperatorTok{$}\NormalTok{units }\OperatorTok{==}\StringTok{ "litre"}\NormalTok{, }\StringTok{"l"}\NormalTok{, df1}\OperatorTok{$}\NormalTok{units)}
\NormalTok{df1}\OperatorTok{$}\NormalTok{units \textless{}{-}}\StringTok{ }\KeywordTok{ifelse}\NormalTok{(df1}\OperatorTok{$}\NormalTok{units }\OperatorTok{==}\StringTok{ "ltrs"}\NormalTok{, }\StringTok{"l"}\NormalTok{, df1}\OperatorTok{$}\NormalTok{units)}
\NormalTok{df1}\OperatorTok{$}\NormalTok{units \textless{}{-}}\StringTok{ }\KeywordTok{ifelse}\NormalTok{(df1}\OperatorTok{$}\NormalTok{units }\OperatorTok{==}\StringTok{ "ltr"}\NormalTok{, }\StringTok{"l"}\NormalTok{, df1}\OperatorTok{$}\NormalTok{units)}
\NormalTok{df1}\OperatorTok{$}\NormalTok{units \textless{}{-}}\StringTok{ }\KeywordTok{ifelse}\NormalTok{(df1}\OperatorTok{$}\NormalTok{units }\OperatorTok{==}\StringTok{ "Litres"}\NormalTok{, }\StringTok{"l"}\NormalTok{, df1}\OperatorTok{$}\NormalTok{units)}
\NormalTok{df1}\OperatorTok{$}\NormalTok{units \textless{}{-}}\StringTok{ }\KeywordTok{ifelse}\NormalTok{(df1}\OperatorTok{$}\NormalTok{units }\OperatorTok{==}\StringTok{ "L"}\NormalTok{, }\StringTok{"l"}\NormalTok{, df1}\OperatorTok{$}\NormalTok{units)}
\NormalTok{df1}\OperatorTok{$}\NormalTok{units \textless{}{-}}\StringTok{ }\KeywordTok{ifelse}\NormalTok{(df1}\OperatorTok{$}\NormalTok{units }\OperatorTok{==}\StringTok{ "gm"}\NormalTok{, }\StringTok{"g"}\NormalTok{, df1}\OperatorTok{$}\NormalTok{units)}
\NormalTok{df1}\OperatorTok{$}\NormalTok{units \textless{}{-}}\StringTok{ }\KeywordTok{ifelse}\NormalTok{(df1}\OperatorTok{$}\NormalTok{units }\OperatorTok{==}\StringTok{ "m"}\NormalTok{, }\StringTok{"ml"}\NormalTok{, df1}\OperatorTok{$}\NormalTok{units)}


\NormalTok{df2 \textless{}{-}}\StringTok{ }\NormalTok{df1 }\OperatorTok{\%\textgreater{}\%}
\StringTok{  }\KeywordTok{mutate}\NormalTok{(}
    \DataTypeTok{amount =} \KeywordTok{case\_when}\NormalTok{(}
\NormalTok{      units }\OperatorTok{==}\StringTok{ "l"} \OperatorTok{\textasciitilde{}}\StringTok{ }\NormalTok{product\_size\_extracted }\OperatorTok{*}\StringTok{ }\DecValTok{1000}\NormalTok{,}
\NormalTok{      units }\OperatorTok{==}\StringTok{ "kg"} \OperatorTok{\textasciitilde{}}\StringTok{ }\NormalTok{product\_size\_extracted }\OperatorTok{*}\StringTok{ }\DecValTok{1000}\NormalTok{,}
      \OtherTok{TRUE} \OperatorTok{\textasciitilde{}}\StringTok{ }\NormalTok{product\_size\_extracted}
\NormalTok{    )}
\NormalTok{    )}

\NormalTok{df2}\OperatorTok{$}\NormalTok{units \textless{}{-}}\StringTok{ }\KeywordTok{ifelse}\NormalTok{(df2}\OperatorTok{$}\NormalTok{units }\OperatorTok{==}\StringTok{ "l"}\NormalTok{, }\StringTok{"ml"}\NormalTok{, df2}\OperatorTok{$}\NormalTok{units)}
\NormalTok{df2}\OperatorTok{$}\NormalTok{units \textless{}{-}}\StringTok{ }\KeywordTok{ifelse}\NormalTok{(df2}\OperatorTok{$}\NormalTok{units }\OperatorTok{==}\StringTok{ "kg"}\NormalTok{, }\StringTok{"g"}\NormalTok{, df2}\OperatorTok{$}\NormalTok{units)}

\NormalTok{data \textless{}{-}}\StringTok{ }\NormalTok{df2 }\OperatorTok{\%\textgreater{}\%}
\StringTok{  }\KeywordTok{select}\NormalTok{(batch,}
\NormalTok{         brand\_name,}
\NormalTok{         manufacturer\_name,}
\NormalTok{         manufacturer\_country,}
\NormalTok{         scan\_country,}
\NormalTok{         amount,}
\NormalTok{         units,}
\NormalTok{         bottle\_count)}

\CommentTok{\#we are left with columns}
\KeywordTok{glue}\NormalTok{(}\StringTok{"We are left with the columns; \{columns\}"}\NormalTok{,}
     \DataTypeTok{columns =} \KeywordTok{glue\_collapse}\NormalTok{(}\KeywordTok{colnames}\NormalTok{(data),}
                             \DataTypeTok{sep =} \StringTok{", "}\NormalTok{,}
                             \DataTypeTok{last =} \StringTok{", and "}\NormalTok{)}
\NormalTok{)}
\end{Highlighting}
\end{Shaded}

\begin{verbatim}
## We are left with the columns; batch, brand_name, manufacturer_name, manufacturer_country, scan_country, amount, units, and bottle_count
\end{verbatim}

\begin{Shaded}
\begin{Highlighting}[]
\CommentTok{\#\#perform cluster analysis}
\end{Highlighting}
\end{Shaded}


\end{document}
